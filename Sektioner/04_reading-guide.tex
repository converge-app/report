\chapter{Læsevejledning}

Der vil flere gange refereres til eksterne dokumenter, såvel som interne dokumenter, der vil kunne findes i dokumentationen. Alle referencer til både interne og eksterne dokumenter vil der henvises til referencelisten sidst i rapporten, og vil refereres med forfatter og udgivelsesår. Den refererede sektion vil være beskrevet på referencen i rapporten, f.eks. “DevOps (Converge-team, 2019, DevOps)”. I eksemplet henvises til et internt dokument, hvor forfatterne er Converge-teamet, hvilket består af medlemmerne af Bachelor gruppen, dokumentet er udgivet i år 2019 og referere til ordet DevOps \cite[DevOps]{converge-terms} i dokumentet for begreber, hvilket står i referencelisten bagerst i rapporten.

Interne såvel som eksterne dokumenter vil blive behandlet ens, hvor referencer både kan bestå af artikler, hjemmesider, opslag osv. De interne dokumenter af Bachelor gruppen vil være markeret med et web-link til det opbevaringssted hvor dokumentet findes, men den afleverede dokumentation kan også bruges. Grunden til dette er at dokumentationen ikke er udgivet andre steder end ved afleveringen, og læseren skal have adgang til de givne dokumenter.

Til selve indholdet, skal det bemærkes at rapporten er opdelt i 3 bider. Den første del er fra Resume, til og med Metode \& proces. Den anden del er fra Teknologiundersøgelse, til og med Tests. Den sidste del er resten af dokumentet. Grunden til at dette dokument er opdelt sådan er grundet læsbarheden. Gruppen ønskede at Del 1 og 3 var læsbare af hvem som helst, dog med hjælp af referencer. Hvor del 2 er skrevet mod en teknisk leder, eller den almene udvikler.

\section{Dokumenter \& Bilag}

Til rapporten følger en række dokumenter, disse dokumenter udgøre bilaget og er beskrevet som dokumentationen. De forskellige dokumenter kan enten findes i det vedhæftede bilag, eller på github under samme titel. Rækkefølgen på dokumenter i bilagget, passer til rækkefølgen af sektioner i rapporten.

Hvert opslag i dokumentation kan findes ved at gå til bilaget ``converge-project/documents/<document-name>''. Det samme gælder for tilhørende source kode, som ligger i mappen ``converge-project/source/<source-project>''.

\begin{itemize}
    \item report
    \item issue-statement
    \item requirments-specification
    \item process-description
    \item cooperation-agreement
    \item research
    \item system-description
    \item software-architecture
    \item software-design
    \item system-interface
    \item deployment
    \item development-environment
    \item security
    \item integration
    \item unit-test
    \item system-tests
    \item integration-test
    \item auto-accepttest
    \item accepttestspecifikation
    \item test-users
    \item bugs
    \item daily-standup
    \item meetings
    \item diagrams
    \item uml-diagrams
\end{itemize}

