\chapter{Arkitektur}
I dette afsnit vil der blive gennemgået en overordnet arkitektur for hele systemet. Afsnittet starter med en gennemgang på  systemniveau og arbejder sig ud til de individuelle moduler såsom klientapplikation og server.  

\subsection{Systemarkitektur}
På nedstående figur ses systemarkitekturen, som er repræsenteret i form af en domænemodel over systemet Converge.

figur(xxx)

figur(xx)viser en oversigt over de muligheder, man kommer til at have som bruger. Dette betyder at Converge systemet
skal kunne håndtere flere forskellige interaktion fra brugeren, samt data. Samtidig er kravspecifikation og domænemodellen
blevet brugt til at finde frem til en række funktionaliteter og hvordan de er afhængige af hinanden. Dette
medfør til at hvordan håndtering af de forskellige funktionaliteter sker, såsom: brugeren, login, signup,dashboard,
chat, indstillinger, portfolio, kategorier, betaling, søgning og filer. Ud fra disse informationer er der blevet udarbejdet
et komponent diagram over komponenter som ses på nedstående figur.

figur (xxx)

figur(xxx) viser et komponentdiagram over det aktuelle system under udvikling i forskellige høje funktionalitetsniveauer.
Hver komponent er ansvarlig for et klart mål inden for hele systemet og interagerer kun med andre væsentlige
elementer på et behov-til-kendskabsbasis. Derudover kan vi se ud fra figur(xxx), at presentations komponenterne
ligger i en samlet pakke og grunden til at de gør det er, fordi det er her brugeren interagerer.

Disse funktionaliteter ligger op til at benytte model-view-controller (MVC) arkitekturen, der står for model, view og
controller. Model repræsenterer business loggik, view viser brugergrænseflade og controller, som er med til at håndterer
brugeranmodningen. I dette tilfælde er MVC brugt på flere måder, både i vores services og med hele vores
system. På serviceniveau bruger vi det i ASP.NET Core, med controllers, models og views (json/swagger). På system
niveau, bruger vi det, som at applikations services er controllers, mongodb som model og React/Next som
View.

figur(xxx)

\subsection{Klientapplikation}

Arkitekturen for klientapplikationen er udarbejdet ud fra domænemodel og user stories i tankerne. Derudover blev gruppens ønske også taget i betragtning om at det skulle give brugeren en god brugeroplevelse og samtidig skulle miljøet være ekstremt nemt og udvikle i. Netop derfor er applikationen udviklet som SPA (Single Page Application), der overholder gruppens kriterier. Herudover fungerer klientapplikationen, som grænsefladen for serveren og grænsefladen for de brugere der benytter applikationen.

figur (xxx)


\subsection{Server applikation}