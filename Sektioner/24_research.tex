\chapter{Teknologiundersøgelse}

I starten af dette projekt blev der udarbejdet en række teknologiundersøgelser, for at se hvilke muligheder der fandtes for at nå det mål der var fastsat til projektet. Projektet omhandler en webapplikation og en webserver, derfor skulle der specifik findes teknologier der egnede sig til at udviklingen af disse \cite{Research}.    

\section{React}

Til at udvikle Webapplikation er der blevet brugt udviklingsplatform React \cite[React]{converge-terms}. React er et JavaScript-bibliotek til opbygning af brugergrænseflader. Vi valgte React fordi der skulle i dette projekt udvikles et dynamisk website. Derudover er det nemt at lære, også er det samtidig enkelt at forstå. Den komponentbaserede tilgang, veldefinerede livscyklus og brug af blot almindelig JavaScript gør React meget let at lære, bruge til at opbygge en webapplikation og at understøtte den. Derudover giver React mulighed for at repræsentere hierarkiske data i en træstruktur, hvilket repræsenterer komponent-baseret design.

\section{ASP.NET Core}

Til at udvikle en webserver, er der behov for at holde styr på interaktionerne på klientapplikationen, derfor blev der besluttet at der var behov for en webserver. I gruppen blev der gjort mange overvejelser over hvilken platform webserveren skulle udvikles i. Der blev dog valgt at webserveren skulle udvikles i ASP.NET Core, da det opfyldte de behov og funktionaliteter der var behov for i dette projekt. ASP.NET Core er en webudviklingsplatform, der giver en omfattende softwareinfrastruktur \cite[Softwareinfrastruktur]{converge-terms} og forskellige tjenester, der kræves for at opbygge robuste webapplikationer. Der er flere gode grunde til at bruge ASP.NET, når man udvikler webapplikationer: høj hastighed, lave omkostninger og enorm online støtte. 

ASP.NET Core er en cross-platform højeffektiv open source framework til opbygning af moderne, cloud-based \cite[Cloud Based]{converge-terms}, internetforbundne applikationer. ASP.NET Core gør det også let at finde online ressourcer. Den primær grund for valget, er at ASP.NET Core er et framework til C\# og det var et ønske fra gruppens side at det skulle være indenfor programmeringssproget C\#. Derudover var gruppen enige om at det skulle være nemt at udvikle services. På denne måde kunne projektets behov opfyldes. Derudover egner ASP.NET Core sig til microservice \cite[Microservice]{converge-terms} udvikling


