\chapter{Teknologiundersøgelse}

I starten af dette projekt blev der udarbejdet en række teknologiundersøgelser, for at se hvilke muligheder der fandtes for at nå det mål der var fastsat til projektet. Efter en smule research, blev der fundet frem til hvilken teknologier der skulle bruges for at opnå målet med projektet. Projektet omhandlede en Web App og derfor skulle der specifik findes teknologier der egnede sig bedste til at udvikle Web Apps.     

\subsection{React}

Til at udvikle klientapplikationen er der blevet brugt udviklingsplatform React. React er et JavaScript-bibliotek til opbygning af brugergrænseflader. Grunden til at valget faldt over React, er fordi der skulle i dette projekt udvikles
et dynamisk website. Derudover er det nemt og lære også er det samtidig enkelt og forstå. Den komponentbaserede tilgang, veldefinerede livscyklus og brug af bare almindelig JavaScript gør React meget let at lære, opbygge en webapplikation og understøtte den. Derudover giver React mulighed for at repræsentere hierarkiske data i en træstruktur.

\subsection{ASP.NET Core}

Til at udvikle en dynamisk applikation, er der behov for at holde styrer på interaktionerne på klientapplikationen, derfor blev der besluttet at der var behov for en server. 
I gruppen blev der gjort mange overvejelser over hvilken platform serveren skulle udvikles i, dog faldt valget over at serveren skulle udvikles i ASP.NET. Da det opfyldt de behov og funktionalitet der var behov for i dette projekt. ASP.NET er en webudviklingsplatform, der giver en omfattende softwareinfrastruktur og forskellige tjenester, der kræves for at opbygge robuste webapplikationer. 
Der er flere gode grunde til at bruge ASP.NET, når man udvikler webapplikationer. Høj hastighed, lave omkostninger og enorm sprogstøtte. ASP.NET er indbygget i det velkendte Windows-servermiljø, hvilket kræver mindre opsætning og konfiguration end andre webudviklingsplatforme, der skal installeres og konfigureres separat. ASP.NETs gør det let at finde online ressourcer. 
Den primær grund for valget, er at ASP.NET er et framework til \C# og det var et ønske fra gruppens side at det skulle være indenfor programmeringssprog C#. Derudover var gruppen enige om at det skulle være nemt og udvikle services. På denne måde kunne projektets behov opfyldes. 


