\chapter{Udeståender}

I dette afsnit vil udeståender blive beskrevet. Opsætning vil være et nummer, en kort beskrivelse og et link til det oprettede problem på GitHub, under den funktionalitet der giver problemer, hvis ikke vil den ligge på Converge-SPA og have et label unknown.

\begin{table}[H]
\centering
\begin{tabular}[c]{p{2cm}|p{5cm}|p{5cm}}
\textbf{Repository / Nummer} & \textbf{Title}                                                            & \textbf{Link}                                             \\\hline
report\#119       & \multicolumn{1}{p{4cm}|}{Add reload after user has created a profile with Stripe}           & \multicolumn{1}{p{5cm}}{https://github.com/converge-app/report/issues/119} \\\hline
report\#121       & \multicolumn{1}{p{4cm}|}{Create pointer for link (Css)}                                     & \multicolumn{1}{p{5cm}}{https://github.com/converge-app/report/issues/121} \\\hline
report\#123       & \multicolumn{1}{p{4cm}|}{Add border around my issues, to show which belongs to user}        & \multicolumn{1}{p{5cm}}{https://github.com/converge-app/report/issues/123} \\\hline
report\#124       & \multicolumn{1}{p{4cm}|}{Add amount to open projects}                                       & \multicolumn{1}{p{5cm}}{https://github.com/converge-app/report/issues/124} \\\hline
report\#117       & \multicolumn{1}{p{4cm}|}{Add checks for userid in sensitive resources (With :sub from JWT)} & \multicolumn{1}{p{5cm}}{https://github.com/converge-app/report/issues/117}
\end{tabular}
\caption{Beskrivelse af udeståender for tilstanden på det nuværende produkt}
\label{tab:bugs}
\end{table}
