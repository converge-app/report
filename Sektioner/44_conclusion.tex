\chapter{Konklusion}
\label{cha:conclusion}

Formålet med dette project var at udvikle en platform, der skulle gøre det muligt at arbejde på en ny måde. Med Converge er det muligt for en udbyder at kunne købe arbejdskraft og ekspertise til et ønskes produkt eller område.

Converge har været udviklet med både slutbrugeren og udvikleren i centrum. For at give udvikleren den bedste mulighed for at lave det bedste produkt. Converge har også været udviklet efter moderne start-up metoder, og har haft fokus på at få et Minimal-Viable-Produkt i luften. Dette har været gjort med moderne teknologier og arbejdsmetoder. Med en decentral struktur drevet af en agil tilgang til software udvikling.

Converge har været designet fra grunden op med fokus på at inddrage brugeren, derfor har rigtige testpersoner været anvendt til at pege produktet i den rigtige retning, samt at kvalitetsikre produktet. Converge, har gået fra design til produktion, og er med få ændringer klar til at rulle en lille smule early adopters ind.

Softwaren har været testet, med unittests, integrationstests samt systemstests og et automatisk accepttest system. Ikke nok til en 100\% fyldesgørende test, men nok til at der er en hvis konfidents i systemet. Accepttest anses som godkendt, med enkelte fejl meldt som udeståender, med tilhørende aktionsplan.

Det kan konkluderes, at projektet er veludført, og det udviklede produkt er fyldesgørende, med de vigtigste krav implementeret for at opnå visionen og løsningen på problemstillingen.

% chapter Konklusion (end)