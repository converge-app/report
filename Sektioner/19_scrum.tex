\subsection{Scrum}

Der er i projektet benyttet Scrum \fxfatal{Reference scrum} som led i en agil udviklingsproces. Scrum er et agilt framework, der kan bruges til at holde overblik over udviklingsforløbet for et projekt. Med Scrum er det muligt at estimere ca. hvor lang tid en funktionalitet vil tage at fremstille, samt holde overblik over hvem laver hvad, og endvidere i hvilket stadie de forskellige opgaver er i. Scrum har nogle ceremonier der består i at medlemmerne af gruppen er blokeret så lidt som muligt og kan arbejde sammen så vidt muligt.

Alle medlemmer af Scrum bruger Sprint boardet som deres primære opslagstavle. På denne tavle er der nogle kolonner til hvert stadie en opgave kan være i, fra backlog, to-do, in progress, blocked, review og closed. I starten af et sprint vil opgaver være placeret i to-do, efter de opgaver der er planlagt er sat ind, så arbejdsmængden passer med den givne sprint længde. I gruppen blev et sprint på 2 uger brugt. Dette er for at medlemmerne kunne nå at få færdiggjort en funktionalitet i dette tidsrum, siden gruppen besluttede tidligt at dette var en god længde. En opgave i Scrum er defineret som en task eller story og indeholder noget nyttigt information, såsom hvad der skal ske, hvilken user story eller arbejdsopgave den tilhører, samt hvor meget tid det regnes med at tage. Efter et par uger, er det muligt for gruppen at estimere hvor meget de kan nå på et sprint. Dette er specielt nyttigt til at estimere hvor meget der kan blive lavet, og hvornår deadlines kan sættes.

I Scrum er der en række forskellige roller. En af disse roller er en Scrum Master \fxfatal{Reference scrum Master}, hvis primære opgave er at styre nogle af de møder der tit bruges til Scrum, såsom estimering, planning, daily scrum meeting osv, samtidig med at holde styr på at Scrum boardet er i fin stand og at det er klart til de næste sprints. I dette tilfælde var det projektlederen som også tog rollen som Scrum Master, da der var en glidende overgang med disse.

En anden rolle i Scrum er Product Owner \fxfatal{Reference Product Owner}, hvis primære opgave er at vedligeholde scrum backlog, så opgaver er klar til estimering, ved refinement og planning. En product owners vigtigste opgave er at kategorisere opgaver efter prioritet, og har normalt kontakt med kunden, for at formidle hvad udviklere arbejder på og hvad der er vigtigst for kunden. I dette tilfælde var hele gruppen Product owners, det har altså været alle medlemmer af gruppen der har opdateret backloggen med nye opgaver samt prioriteringen af disse.

Den sidste rolle er udviklingsteamet, som alle gruppens medlemmer har været en del af. Inklusivt Scrum Masteren som har været et fungerende medlem. Udviklingsteamets fokus ligger på at få lavet de opgaver der ligger i to-do til closed. Deres opgave er også at deltage i Scrum møder og give deres ekspertise til at give opgaver et estimat - noget der måske er for teknisk for en Product Owner.