
\section{Server applikation}

Sammenhængen mellem Web App og Webserver kan beskrives som en MVC løsning, hvor Web applikationen fungere som view og Webserveren som Model-Controller. Dog er dette en grov simplificering af hvordan resultatet skal være. Web Applikationen skal kommunikere med Webserveren, som ikke kun er en enkelt server, men en samling af dem som arbejder parallelt, kaldet et cluster. Dette cluster fungere som en webserver, og Web applikationen ved ikke bedre.

Webserveren har en primær funktion: At behandle anmodninger med passende forretningslogik. Adgangen til denne forretningslogik foregår igennem forskellige addresser på Webserveren. Disse addresser udegår den udstillede Webserver API. Det skal siges at Webserveren og Converge-Cluster er andet til samme formål, dog hvor Webserveren beskriver selve ideen fra et overbliks perspektiv, og Converge-Cluster for det egentlige produkt. 

Derudover vil Webserveren ved behandling af anmodninger gemme resultater i et persisterings lag, enten som filer eller rækker i en database. Dette information skal kunne hentes til den funktion der har brugt for det, og som har adgang. De forskellige services (selvstændige applikationer) i Webserveren, så er der $n$ lag: Et samlingslag, Et præsentationslag (API), et forretningslag (forretningslogik), og et dataadgangslag. Disse tre lag placeres i en teknologisk sammenhæng, hvor ASP.NET Core platformen er valgt. Denne platform har programmerbar routing, der samler alle kald til de individuelle services.

\fxfatal{Insert figure of the Webserver layer with a reverse proxy in front of a service}

Resten af afsnittet indeholder uddrag fra dokumentation \fxfatal{Reference documentation architecture}.

\textbf{Samlingslaget} samler de underlæggende applikationer under en udgang, dette gør det muligt for Web App'en at nemt kunne skifte mellem de forskellige services og endpoints. Samlingslaget må ikke indeholde forretningslogik, og skal bare beskytte de bagvedlæggende services.

\textbf{Præsentationslaget} præsenterer et JSON API til omverdenen. Dette API forventes at modtage JSON indpakket data ved anmodninger, og ingen data ved forespørgsler. Præsentationslagget er tyndt og indeholder kun lige nok data til at verificiere indholdet, samt at verificere klienters anmodningsrettigheder. Det forventes at hvis en anmodning er valideret, vil den blive sendt videre til forretningslaget, hvilket kan håndtere dette. Alt i alt vil det sige at data kan modtages og returneres i dette lag, men kun det overfladiske er valideret, resten er bestemt i forretningslaget.

\textbf{Forretningslaget} indeholder services, der agere på klienters forespørgsler i systemets domæne, her bliver forespørgsler håndteret baseret på forretningslogik. Dette kan f.eks. være kommunikation til andre services, for at få deres resourcer eller videresende begivender til interesserede parter. 

\textbf{Dataadgangslaget} er bygget op omkring 2 forskellige services, enten en Http klient eller en databaseadgangsklient. Til en httpklient er der brugt et fælles bibliotek der kan tilgå systemets resourcer, dette gør at to forskellige services kan tale sammen, selvom man skriver C\# kode. Til en databaseadgangsklient er der brugt repositories, ved et repository beskriver man hvordan databasen skal tilgåes. Repositories udgiver en grænseflade som er implementeret med det klient kode der skal til for at kommunikere med database. I dette tilfælde er MongoDBClient brugt, pga. database typen som er mongodb.

Server applikationen skal agere som en fælles front, men have forskellige komponenter i maven. Nogle som præsentere et nemt at bruge grænseflade, og andre lidt mere komplekse som bruges inde i selve systemet. I dette tilfælde kan det f.eks. være til Users Service. Hvilket er en af byggestenene for systemet, det beskriver noget om brugeren og indeholder et id som bruges i næsten alle andre services. Users service bliver f.eks. brugt af authentication service som skal udstede et API til registrering og login. Dvs. at Users service primært er længere nede i hirakiet end authentication service.