\chapter{Diskussion}
\label{cha:discussion}

Projektet havde til formål at skabe et moderne arbejdes platform ind i det 21. århundrede, der kunne bruges af bruger der ønsker at enten arbejde selvstændig med projekter og have frihed under eget ansvar eller få løst kort eller langvarigt projekter, efter deres ønske. På denne måde skulle det være muligt at både få løst og tage projekter, uanset hvor brugeren befandt sig i verden. Det endelige resultat endte ud med en web platform, hvor det består af en front-end, back-end og en database.  Front-enden er udarbejdet som SPA (single page application). Derudover er front-enden er serverløs det betyder at man ikke har en server som man skal holde styr på eller kører, men det gør andre såsom: Google cloud. Back-enden er udarbejdet i ASP.Net core og databasen er mongodb. Herudover består systemet af en række værktøjer (Docker, Kubernetes, Helm, Jaeger, Traefik osv.), der gør udvikling nemt og samtidig har det gjort at målet er nået. 

Her kan der diskuteres om alle mål og forventninger er opfyldt for webapplikationen Converge. Ud fra gruppens ambitiøse mål og forventninger er alle målene opfyldt og derfor fungere efter hensigten. Dog skal det siges at der er fortaget nedskæringer i funktionalitet, hvor gruppen var enig om hvilken funktionalitet der var vigtige for en prototype. For eksempel kan en bruger ikke få tilsendt en ny kode, logge ind med google, samt andre små funktionaliteter. Disse funktionaliteter kan brugeren godt undvære i første om gang, dog med tanker om at tilføje dem, hvis der videreudvikles på systemet. 

En anden mål som gruppen havde for systemet, var at benytte forskellige værktøjer til at gør udviklingen nemmer og samtidig opnå målet. Disse værktøjer har haft hver deres formål og har dermed opfyldt gruppens behov på hver deres måde. For eksempel er Docker, Kubernetes, Helm blevet brugt til udrulning af produktionsmiljøet, Jaeger er blevet brugt som spores værktøj, der finder spore mellem de forskellige relationer som service har og der findes yderligere værktøjer der er blevet benyttet til udvikling af Converge platformen afsnit(xxx). 

% chapter Diskussion (end)