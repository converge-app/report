\section{Server applikation}
Server applikationen består af en række API’er, som bruges til at lagrer og udlevere data til brugeren. De opbevarede data kan webbrowsere hente via protokollen HTTP eller HTTPS. 
Converge sytemet benytter disse API’er benyttes til at komunikere med serveren. Den måde det fungerer på er at en klient sender en HTTP-anmodning til serveren, serveren returnerer et svar til klienten. svaret kan indeholde oplysninger om den anmodning brugeren ønsker. På nedstående figur ses kommuninkationen mellem brugeren og serveren.

\fxfatal{Missing figure}

Ud fra figuren ses at der bliver benyttet en serverløs til kommuninkationen med serveren. Som det første sender en client en anmodning og anmodningen bliver modtaget i API-gateway. API-gateway er lavet som function-as-a-Service (FaaS), hvor grundlæggende handler det om at kører backend-kode uden at styre dine egne serversystemer.
Derudover fungerer en API-gateway som en HTTP-server, hvor ruter og slutpunkter er defineret i konfigurationen, og hver rute er knyttet til en ressource til at håndtere denne rute. Når en API-gateway modtager en anmodning, finder den routingskonfigurationen, der matcher anmodningen. Typisk tillader API-gatewayen at kortlægge fra HTTP-anmodningsparametre til et mere kortfattet input til FaaS-funktionen, eller tillader, at hele HTTP-anmodningen passeres, typisk som et JSON-objekt. FaaS-funktionen udfører sin logik og returnerer et resultat til API-gatewayen, som igen vil omdanne dette resultat til et HTTP-Response, som det vender tilbage klienten.


