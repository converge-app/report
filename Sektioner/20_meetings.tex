\subsection{Møder}

Der har i gruppen været forskellige typer møder, herunder interne og eksterne møder. De interne møder har været imellem gruppensmedlemmer, og har været forbundet med Scrum, til eksempelvis refinement, planning eller retrospectives. Normalt har der også været arbejdsmøder, hvor gruppen har siddet og arbejdet i samme lokale for at sparre. De eksterne møder er møder hvor andre en gruppensmedlemmer har deltaget, om det er vejleder eller testpersoner. Gruppen har ugentligt haft et vejledermøde med gruppens vejleder, hvor et kort resume, og nogle gange en Sprint demo blevet vist, samt spørgsmål stillet og vejleder opdatered med status på projektet.

Testperson møder er også blevet holdt, Projektlederen har flere gange haft møder med eksterne parter, for at få deres indblik og ekspertise på hvordan en bruger vil bruge systemet. I løbet af hele projektet har testpersoner deltaget undervejs i flere stadier af produktets udvikling. Disse møder har tit været på tidspunkter hvor den givne testperson kunne, så har været om aftenen i hverdage, eller i weekenden. Disse har som regel varet i 1-2 timer og har konkluderet i en mundligt præsentation for gruppen, såvel som et skriftligt dokument, med de vigtiste pointer fra test personen.

Til alle vejledermøder har der været en Mødeleder samt en referent. Det har været mødelederens ansvar at få planlagt mødet og udsendt en mødeindkaldelse til de personer, der skulle deltage. Gruppens projektleder tog naturligt denne rolle. Mødereferater og mødeindkaldelser kan findes på \fxfatal{Reference gitbooks meetings}.