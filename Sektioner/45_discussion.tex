\chapter{Diskussion}
\label{cha:discussion}

Projektet havde til formål at skabe en moderne arbejdsplatform ind i det 21. århundrede, der kunne bruges af brugere der ønsker at enten arbejde selvstændigt med projekter og have frihed under eget ansvar, eller få løst kort- eller langvarige projekter, efter eget ønske. På denne måde skulle det være muligt at få både løst og modtaget projekter, uanset hvor brugeren befandt sig i verden. Det endelige resultat endte ud med en web platform, som består af en front-end, back-end og en database. Front-enden er udarbejdet som SPA (single page application). Derudover er front-enden serverløs, hvilket betyder at man ikke har en server som skal holdes styr på eller køre, men det gør andre såsom Google cloud. Back-enden er udarbejdet i ASP.Net core og databasen er MongoDB. Herudover består systemet af en række værktøjer (Docker, Kubernetes, Helm, Jaeger, Traefik osv.), der gør udvikling nemt og samtidig har det gjort at målet er nået.

Her kan der diskuteres om alle mål og forventninger er opfyldt for webapplikationen Converge. Ud fra gruppens ambitiøse mål og forventninger er alle målene opfyldt og derfor fungerer efter hensigten. Dog skal det nævnes at der er fortaget nedskæringer i funktionalitet, hvor gruppen var enig om hvilken funktionalitet der var vigtige for en prototype. For eksempel kan en bruger ikke få tilsendt en ny kode, logge ind med google, samt andre små funktionaliteter. Disse funktionaliteter kan brugeren godt undvære i første omgang, dog med tanker om at tilføje dem, hvis der videreudvikles på systemet. 

Et anden mål som gruppen havde for systemet, var at benytte forskellige værktøjer til at gøre udviklingen nemmere og samtidig opnå målet. Disse værktøjer har haft hver deres formål og har dermed opfyldt gruppens behov på hver deres måde. For eksempel er Docker, Kubernetes, Helm blevet brugt til udrulning af produktionsmiljøet, Jaeger er blevet brugt som sporingsværktøj, der finder spor mellem de forskellige relationer som service har og der findes yderligere værktøjer der er blevet benyttet til udvikling af Converge platformen i dokumentationen \cite{Research}.

Under udviklingen blev det taget i betragtning at dette produkt ville få konsekvenser for samfundet. For det første er det en arbejdsform som er under kontrovers, da en af præmisserne er at freelancers er selvstændige. Men dette gør også at de ikke nødvendigvis har de samme fordele som ved at arbejde for et større firma, f.eks. med sygesikring, betalt ferie, osv. Samtidig skal der også indføres kontroller, for at finde ud af om brugerne anvender systemet til at lave hvidvask, samt generelt adfærd der går imod terms of use.

I forhold til andre lignende platforme, skiller Converge sig på nuværende tidspunkt ikke meget ud. Grunden til dette er at video funktionaliteten ikke er implementeret. Grundet at en bruger stadig kan kommunikere med andre brugere over tekst chat, om det er på selve collaboration eller via. chat funktionalitet. Beslutningen blev taget ud fra det givne kompleksistets niveau, hvis videochat skulle implementeres ville de estimerede point fylde et halvt bachelor projekt, dette var ikke en prioritet for bachelor gruppen, eftersom målet var at lave en multifunktionel platform ved brug af Agile metoder, og ikke en næsten en-dimensionelt produkt med inkluderet video-chat. Gruppen valgte at fokuserer på et produkt, som i virkeligheden kunne udrullet hvornår som helst, med mindre ændringer.

Converge blev fremstillet via. en Cloud-native tilgang, efter gruppen havde arbejdet et stykke tid på applikationen, kunne det med den nuværende erfaring siges at det tager længere tid at udvikle en distribureret platform end en monolith, hvor al funktionalitet er i samme applikation. Gruppen valgte at bruge en Cloud-Native platform, eftersom det giver bedst mulighed for størrere virksomheder at arbejde parallelt, og opdele ansvar. Dette har dog ikke været nødvendigt for bachelor-gruppen. Derfor kan det siges at Converge har brede udviklingsmuligheder, men kræver store investeringer for at få udrullet det komplette produkt. Men for Bachelor gruppen har det været vigtigt, eftersom Cloud-Native udvikling er brugt af næsten alle mellemstore til store virksomheder.

% chapter Diskussion (end)