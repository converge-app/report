\section{Domæne analyse}

Til projektet er der opsat \emph{personaer} \cite[Personaer]{converge-terms}, som en afbildning af den målgruppe Converge projektet vil fokusere på, og tænker som de primære kunder.

Personerne er opdelt i 2 kategorier, slutbrugere og personale. Slutbrugere er kategoriseret som brugere, der hvor almene funktioner er gældende: disse funktioner bliver delt i mellem de to slags slutbrugere, freelancers og employeers/firma. Til at holde fokus på de forskellige brugere under udviklingen er der brugt personaer, til at give en fiktiv historie til en bruger som kunne være interesseret i systemet. Der er oprettet en til flere personaer, for hver aktør \cite[Aktør]{converge-terms} og hver type person af målgruppen.

\begin{table}[H]
    \begin{small}
        \caption{Aktør beskrivelse for employer}
        \label{tab:employer}
        \begin{center}
            \begin{tabular}[c]{p{3cm}|p{8cm}}
                \multicolumn{1}{c|}{\textbf{Navn}} & \multicolumn{1}{c}{\textbf{Employer}}                                                                                                                                                                                           \\
                \hline
                Beskrivelse                        & \multicolumn{1}{p{10cm}}{Employer er en bruger som ønsker at få lavet et stykke arbejde som han/hende ønsker gjort. Personen søger den ekspertise der er på platformen, og derfor nogle af de freelancers der befinder sig.}    \\
                \hline
                Persona                            & \multicolumn{1}{p{10cm}}{Jens er en entrepeneur på 40 år, han har sit eget murefirma og har brug for en ny brochure til sit firma, hans budget er ikke stort, men han vil stadig gerne have god kvalitet.}                      \\
                                                   & \multicolumn{1}{p{10cm}}{Karsten er ansat i en størrer design virksomhed, denne virksomhed bruger som regel outsourcing til at fremstille websites fra deres designs. Han er den primære person ansat til at styre outsourcing} \\
                \hline
                Mål                                & \multicolumn{1}{p{10cm}}{At udnytte Converge til at finde god og billig hjælp til opgaver som kræver ekstra hænder.}                                                                                                            \\
            \end{tabular}
        \end{center}
    \end{small}
\end{table}

\begin{table}[H]
    \begin{small}
        \caption{Aktør beskrivelse for freelancer}
        \label{tab:freelancer}
        \begin{center}
            \begin{tabular}[c]{p{3cm}|p{8cm}}
                \multicolumn{1}{c|}{\textbf{Navn}} & \multicolumn{1}{c}{\textbf{Freelancer}}                                                                                                                                                                            \\
                \hline
                Beskrivelse                        & \multicolumn{1}{p{10cm}}{Freelancer er en bruger som ønsker at bruge Converge til at tjene penge ved brug af sin ekspertise}                                                                                       \\
                \hline
                Persona                            & \multicolumn{1}{p{10cm}}{Mette er en Web designer som bruger Converge ved siden af sit fuldtidsjob. Mette har brug for nogle ekstra penge, så hun tager gerne mange clienter og er en flittig bruger af systemet.} \\
                \hline
                Mål                                & \multicolumn{1}{p{10cm}}{At udnytte Converge til at finde klienter, både kortvarigt og langsigtet.}                                                                                                                \\
            \end{tabular}
        \end{center}
    \end{small}
\end{table}

For resten af aktører og personaer henvises til dokumentation ~\cite{documentation-kravspec}

Til den anden kategori er personale, dette er inddelt i aktørerne, developer, supporter og Administrator \cite[Personale]{converge-terms}. Disse aktører repræsenterer det interne personale i Converge-teamet og vil blive behandlet som var det en slutbruger. For flere detaljer for personalet se dokumentationen \cite[Funktionelle krav]{documentation-kravspec}. 
